% \usepackage{hyperref}

\newcommand{\OwnerDefinitionFootnote}{%
    \textbf{The Office of the Federal Register}:
    \textsl{Title 49 -- {\S} 376.2(d)} \\
    \href
        {https://www.ecfr.gov/current/title-49/part-376/section-376.2\#p-376.2(d)}
        {https://www.ecfr.gov/current/title-49/part-376/section-376.2}
}

\noindent%
This AGREEMENT (``Agreement''), made as of this \textbf{\today}, by and
between \textbf{\CarrierName} (``Carrier'') and {\xrfill[-1pt]{0.5pt}}
(``Contractor'').

\section{Recitals}

\begin{enumerate}
    \item The Carrier holds permits under authority granted by federal,
    state, and other agencies. Carrier is authorized by the Federal Highway
    Administration, with
    {\bfseries {\scshape MC Number} -- \CarrierMC}
    and
    {\bfseries {\scshape US DOT Number -- \CarrierUSDOT}},
    to provide transportation of property under contract with brokers,
    shippers, and receivers of general commodities. This service is
    facilitated through arrangements with independent contractors
    possessing motor vehicle equipment.
    
    \item The Contractor is the owner of the tractor and/or trailer
    equipment described in the
    \hyperref[subsection:equipment-description]{Equipment Description}
    section and is duly authorized and empowered to execute this Agreement.
    The Contractor is engaged in the business of transporting freight by
    motor vehicle on behalf of and pursuant to independent contractor
    agreements with for-hire motor carriers and shippers. As used herein,
    "Contractor" means any person defined in 49 CFR {\S} 376.2(d)%
    \footnote{\OwnerDefinitionFootnote}, including the Contractor and its
    drivers, employees, agents, or independent contractors.
    %
    \subimport{equipment/}{index.tex}

    \item Carrier and Contractor desire to enter into an agreement to carry
    out the foregoing.
\end{enumerate}