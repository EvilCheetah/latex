{
    \renewcommand
        {\SecondLevelEnumerator}
        {%
            {\SectionName}:
            \textsl{Equipment} -- (\theenumii)%
        }


    \underline{Equipment}. Contractor agrees and warrants:
    \begin{enumerate}[
        ref = \SecondLevelEnumerator
    ]
        \item To furnish the equipment (``Equipment'') identified in the
        Equipment Description section herein. The Contractor's Equipment
        list may be amended by an amendment signed by authorized
        representatives of Carrier and Contractor with copies provided to
        each.

        \item That equipment subject to this Agreement meets the US
        Department of Transportation (``DOT'') safety requirements and
        standards. Carrier reserves the right to terminate this Agreement
        at any time that Equipment subject to this Agreement fails to meet
        DOT safety requirements and standards.

        \item That Contractor holds full legal title or has the legal right
        to exercise full control over the Equipment.

        \item{\label{terms:equipment-identification}} To affix to the
        Equipment the identification provided by Carrier required by
        federal, state, and other agencies, including Carrier's name and MC
        number. Such identification shall be removed immediately and
        returned to Carrier upon termination of this Agreement. In the
        event such identification cannot be removed but has to be painted
        over, Contractor shall immediately cause to have such repainting
        done and shall provide Carrier with a photograph of each repainted
        door, and in each photograph, the front page of a current newspaper
        evidencing the date of the photograph. Failure to furnish evidence
        of the removal of identification from the Equipment will result in
        the withholding of the final settlement.

        \item To pay all operating expenses, including, but not limited to,
        all expenses of fuel (including reefer fuel), oil, tires,
        maintenance, repairs, and other items necessary for the safe and
        efficient operation of the Equipment, meals and lodging,
        installation and usage fees of any and all communication equipment,
        such as Qualcomm or Citizens' Band (CB) radio on the Equipment,
        road taxes, mileage taxes, tolls, ferries, fines for parking,
        moving or weight violations, licenses, permits, or any other
        assessments based upon the operation of the Equipment in compliance
        with the rules and regulations of the DOT and all other federal,
        state, and foreign regulatory bodies having jurisdiction over
        Contractor's operations and Equipment, including all such costs
        related to all loaded and empty mileage. Contractor hereby agrees
        that, if it becomes necessary for Carrier to pay for any of
        Contractor's operating expenses, Carrier shall deduct such cost
        from Contractor's payment upon settlement.
        
        \item To maintain the Equipment in proper operating condition to
        satisfy any federal, state, or Carrier safety inspection at any
        time and to make such repairs as may be found necessary. The
        Contractor shall be responsible for undertaking and passing annual
        DOT safety inspections. The Contractor shall cause all vehicles to
        be in compliance with any and all applicable federal and state
        regulations as presently in force or enacted in the future and
        cause all vehicles to be operated in compliance therewith. The
        choice of location and persons to perform any necessary maintenance
        or repairs is vested exclusively in the Contractor. Contractor
        shall provide Carrier with legible copies of all repairs and
        maintenance performed on the Equipment at the time such work
        records covering the Equipment required by the regulations of the
        DOT and any other applicable governmental agency shall be forwarded
        by Contractor to Carrier by the last day of each calendar month
        during the term of this Agreement. If such records are not
        delivered to the Carrier within 7 days following the end of each
        month, or are incomplete, Carrier reserves the right to terminate
        this Agreement or suspend the same until complete records are
        supplied. Contractor further agrees that Carrier will have no
        obligation or responsibility for the repair of damage to the
        Equipment, regardless of cause.

        \item To ensure that any and all reefer trailers (``Reefer
        Equipment'') subject to this Agreement are calibrated every 3
        months. The Contractor hereby agrees to operate such Reefer
        Equipment under ``continuous operation'' setting at all times while
        transporting freight dispatched by Carrier. For any and all produce
        shipments, Contractor shall use a temperature recorder to create a
        continuous temperature record and to ensure the constant setting of
        the set temperature. Unless said temperature recorder has been
        provided by the shipper, Contractor shall be responsible for the
        cost and maintenance of such temperature recorder.

        \item To conduct a ``Clean Operation'' in compliance with Carrier
        demands. Clean Operation shall include: all items required under
        safety procedures, check calls, on-time loading, on-time delivery,
        deliveries at correct destinations, prompt reporting of all cargo
        discrepancies, proper disposition of refused freight as directed by
        Carrier, no unauthorized handling of freight, no adverse safety
        reports, and no discourteousness while under dispatch or at Carrier
        facility. Carrier shall have the right to terminate this Agreement
        for cause, immediately and without prior notice, upon the occasion
        of any failure to comply herewith.

        \item The Contractor is responsible for ensuring seal integrity
        prior to the departure of a shipment and throughout delivery until
        the entire shipment has been delivered. Contractor is also
        responsible for keeping a continuous seal record, which shall be
        maintained and documented on all bills of lading for each stop of
        the shipment. Incidents where tampering may have compromised the
        integrity of a seal prior to delivery shall be reported to the
        Carrier immediately. Contractor shall not break or otherwise tamper
        with a seal, until and unless the receiver accepts the cargo and
        signs off on the bill of lading and directs the Contractor to break
        the seal. In the event that the Contractor breaks or tampers with a
        seal, or otherwise fails to ensure seal integrity at any point
        prior to delivery, the Carrier shall deem such shipment rejected by
        the receiver and reserves the right to deny payment of the
        settlement to the Contractor.

        \item That in the event the Carrier supplies the trailer to be
        pulled by the Contractor, the Carrier will maintain or cause to be
        maintained the trailer and accessory trailer equipment furnished by
        the Carrier. The Contractor shall be responsible and liable for all
        loss of or damage to any such trailer and accessory equipment
        occurring while in the possession or custody of the Contractor,
        arising out of the negligent or intentional acts of Contractor or
        Contractor's employees or agents, excepting such loss or damage as
        Carrier is compensated for under any applicable insurance or
        otherwise, and Contractor will indemnify and defend Carrier against
        any loss on account thereof. In the event Contractor supplies the
        trailer equipment under this Agreement, Contractor shall be
        responsible for the maintenance of the trailer and accessory
        equipment furnished by Contractor.
    \end{enumerate}
}