\documentclass{exam}

\usepackage{import}

\import{../common/config}{usepackage.tex}


\begin{document}
    \noindent
    \question{Question 1:} For each pair of functions below, first
    determine which function grows asymptotically faster, and then express
    the relation between the two functions using all the asymptotic
    notations (%
        $\boldsymbol{\theta}$,
        $\boldsymbol{   O  }$,
        $\boldsymbol{\Omega}$,
        $\boldsymbol{   o  }$,
        and
        $\boldsymbol{\omega}$%
    )
    that apply. Only give the relations in which $\boldsymbol{f(n)}$
    appears on the left-hand size. \textbf{Justify your answer}, and
    use the limit definition if necessary.\\
    {
        \indent (1)
        \indent $f(n) = n^2 + (log(n))^3$
        \indent $g(n) = n^2 log(n)$
        \indent No justification is required
    }
    \vspace{1in} \\
    {
        \indent (2)
        \indent $f(n) = 4^n$
        \indent $g(n) = 3^n + n$
        \indent Justify your answer using the limit definition.
        Show your work.
    }
    \\
    \\
    \noindent
    \question{Question 2:} If we know that the running time $T(n)$ of
    some algorithm satisfies the relations:
    $\boldsymbol{T(n) = o(n^{2.5} \cdot log(n))}$
    and
    $\boldsymbol{T(n) = \Omega( n \cdot (log(n))^3 )}$
    Which of the following functions can $\boldsymbol{T(n)}$ possibly
    be? Circle \textbu{all} that apply.\\
    {
        \indent
        $n^{2.5} \cdot log(n)$
        \hfill
        $n \cdot (log(n))^3$
        \hfill
        $n \cdot (log(n))^5$
        \hfill
        $n^{1.01}$
        \hfill
        $n^{2.6}$
        \indent
    }
\end{document}